\documentclass[11pt]{ctexart}
\usepackage[hustc]{cnlogo}
\usepackage[T1]{fontenc}
\newcommand\fontnamesong{Source Han Serif SC}
\newcommand\fontnamehei{Source Han Sans SC}
\setCJKmainfont[Mapping = fullwidth-stop,BoldFont=\fontnamehei]{\fontnamesong}
\usepackage{fontspec}
\setmainfont{Times New Roman}
\punctstyle{kaiming}
\usepackage{graphicx}
\usepackage{geometry}
\geometry{
  a4paper,
  top=25.4mm, bottom=25.4mm,
  headheight=2.17cm,
  headsep=4mm,
  footskip=8mm
}
\RequirePackage{setspace}
\setstretch{1.38}
\RequirePackage{xcolor-material}
\usepackage{hyperref}
\hypersetup{
  pdftitle={华中科技大学 LaTeX 版校徽校名},
  pdfauthor={MatNoble},
  colorlinks=true,
  linkcolor=GoogleRed,
  urlcolor=GoogleBlue,
  filecolor=GoogleGreen
}
\pagestyle{plain}
\pagenumbering{roman}
\title{华中科技大学 \LaTeX{} 版校徽校名 \\ --- 基于 cnlogo 宏包 }
\author{\href{https://matnoble.me/about/}{MatNoble}}
\begin{document}
\maketitle

\href{https://github.com/yuxtech/cnlogo}{cnlogo 宏包}是由\href{https://yuxtech.github.io/}{向禹老师}开发制作,旨在:利用 tikz 绘制国内高校的校徽。并配有\href{https://github.com/yuxtech/cnlogo/blob/master/ceshi.pdf}{使用文档}

然而,在使用过程中,发现 cnlogo 宏包中的华中科技大学徽标配色与\href{https://imgkr.cn-bj.ufileos.com/db848d98-3260-44a4-84c3-c6fbe1e7914a.png}{官方配色}不符,而且徽标边缘有锯齿。一番小改动后,使得配色与官方一致,更新矢量图使绘制更清晰,且徽标缩放更容易实现!

{ \centering \includegraphics[width = .8\textwidth]{images/official.png}
  
}

因为``继承''自 cnlogo 宏包,所以使用方法与 cnlogo 宏包相同(也可将宏包安装至电脑本地):将 cnlogo.sty 和 cnlogo 文件夹放入 .tex 文件根目录,在头文件导入 cnlogo 宏包,在正文分别使用 $\backslash$hustclogo、$\backslash$hustctext 和 $\backslash$hustcwhole 命令就可以分别插入校徽、校名和校徽$+$ 校名,比如

\begin{verbatim}
\documentclass{article}
\usepackage[hustc]{cnlogo}
\begin{document}
\hustclogo
\end{document}
\end{verbatim}
即可插入校徽。

在 $\backslash$hustclogo[scale] 方括号中可选参数是缩放比例,默认是 $1.0$。

\clearpage

\subsection*{官方配色的预览:}

\begin{enumerate}
    \item 插入校徽:
    \begin{verbatim}
    {\centering  \hustclogo[0.35]
    
    }
    \end{verbatim}
    {\centering  \hustclogo[0.35]
    
    }
    \item 插入校名:
    \begin{verbatim}
    {\centering  \hustctext[0.35]
    
    }
    \end{verbatim}
    {\centering \hustctext[0.35]
    
    }
    \item 插入校徽 $+$ 校名:
    \begin{verbatim}
    {\centering  \hustcwhole[0.25]
    
    }
    \end{verbatim}
    {\centering \hustcwhole[0.25]
    
    }
\end{enumerate}

\subsection*{纯色徽标预览}

$\backslash$hustclogo[scale] 后可再加 $3$ 个可选方框,根据官方提供 RGB 设定:蓝色(husts1)、红色(hustc2)和灰色(hustc3)。

\begin{itemize}
    \item 校徽蓝
    \begin{verbatim}
    {\centering  \hustclogo[0.3][hustc1][hustc1][hustc1]
    
    }
    \end{verbatim}
    {\centering \hustclogo[0.3][hustc1][hustc1][hustc1]
    
    }
    \item 校徽红
    \begin{verbatim}
    {\centering  \hustclogo[0.3][hustc1][hustc1][hustc1]
    
    }
    \end{verbatim}
    {\centering \hustclogo[0.3][hustc2][hustc2][hustc2]
    
    }
    \item 自定义
    \definecolor{hustb}{RGB}{37,85,122}
    宏包支持根据 RGB 自定义颜色,比如``华科蓝''
    \begin{verbatim}
    \definecolor{hustb}{RGB}{37,85,122}
    {\centering  \hustctext[0.35][hustb]
    
    }
    \end{verbatim}
    {\centering \hustctext[0.35][hustb]
    
    }
\end{itemize}

\noindent\rule[0.25\baselineskip]{\textwidth}{0.9pt}

由于本子项目仍需完善,所以暂时没有 PR 到向老师的 GitHub 仓库。希望看到本项目的 Huster 们可以提出宝贵建议。

\end{document}